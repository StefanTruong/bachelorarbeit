\chapter{Introduction}

The concept of cellular automata (CA) was first introduced by the mathematician John von Neumann\cite{Neumann} in the 1940s. Its popularity increased in the 1970s through its use in Conway's Game of Life, presented by Martin Gardner in the magazine Scientific American \cite{10.2307/24927642}. In general, CA models are represented in a 2D grid where space and time are divided into discrete cells and time steps. After initialization, a cell exchanges information and takes action with neighboring cells within a time step. The model's initialization and transition rules depend on its use in a domain and the author's discretion. Some CA models, such as Rule 110 in the Game of Life, have even been shown to be Turing-complete. CA models have gained popularity in a number of fields because of their ability to simulate complex, emergent phenomena based on simple assumptions. This simplicity has led to the widespread use of CA models in simulations in fields such as biology, economics, geography and computer science.


The aim of this bachelor thesis is to perform an exploratory analysis of the behavior of a platoon of motorcyclists in mixed traffic on a two-lane road for one direction with periodic boundary conditions using cellular automata theory. The platoon in this study is assumed to prefer to drive in a line at a preferred speed based on the curvature of the road, while other vehicles are expected to move forward as fast as possible. Chapter \ref{chapter:Literature} reviews the relevant literature, while chapter \ref{chapter:A Cellular Automata Model for Platoons} presents the model development, parameters and metrics. The definition of road curvature and a fun metric for platoon members are introduced. The relationship between traffic flow and vehicle density on the road will then be explored with this model setup in chapter \ref{chapter:Influence of model parameters on metrics}. Finally, in chapter \ref{chapter:Analysis of platoon behaviour in the Black Forest} the model is applied to curvature data of seven roads in the Black Forest region obtained from \hyperlink{https://roadcurvature.com/}{roadcurvature.com} to investigate the platoon behavior of five motorcyclists in both free and congested traffic conditions. Additionally a fun metric is used to test the ability of the model to distinguish between roads with different levels of curvature in the Black Forest region. The analysis is based on the assumption that road curvature is an important factor in motorcyclists' choice of destination in the Black Forest region.



