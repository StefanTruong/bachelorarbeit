\chapter{Conclusion}
\label{chaper:Conclusion}
The bachelor thesis faced several major problems, including the difficulty of gathering real data without GPS data and a deep understanding of vehicle behavior to make the model more realistic. The lack of reference data means that many parameters, such as when a street is congested or how fast one would like to drive, rely on visual evaluation from the simulation. The data gathered from the website "www.curvature.com" also presented challenges, as a deeper understanding of the Keyhole Markup Language and its handling as a data model in a geographic information system is required. Additionally, the data which was gathered from the tooltips instead are very inaccurate and does not provide information on how to evaluate the beauty of the trip besides its curvature. 

The fun metric used in this thesis is entirely invented and does not reflect other factors that could influence the driving behavior of a platoon as a whole or a motorcyclist individually. Additionally, modeling motorcyclists with different but fixed preferences regarding distances to each other and speed could lead to a problem where incompatible motorcyclists could create a matching problem like that seen in a domino game. Modeling different preferences that would result in a steady formation where all motorcyclists are satisfied with the distance to each other would be like solving a domino game, which is a NP-hard problem. This driving behavior would differ too greatly from realistic behavior. Feedback effects where motorcyclists take decisions from other vehicles were completely left out as well as their modeling complexity would be too great, and there is no literature available on this topic found.

The attempt to model the driving behavior solely according to the motorcyclists' preferences as a normal distribution was abandoned due to coding difficulties and logical inconsistencies. For example, when a motorcyclist is in the middle and wants to drive slower so that the motorcyclist behind can catch up, this could conflict with their own desire to catch up to the motorcyclist in front of them. Feedback effects were not implemented due to their modeling complexity. Additionally, modeling the leader like any other member of the platoon showed unrealistic results, as this simulation suggested that the leader has the lowest fun, which is opposite to reality. The escape velocity, which is an important factor in determining the speed of the platoon, and the compression of the platoon were not researched. In the simulation, all members drove at the same average velocity during the trip making an analysis on the compression of the platoon difficult. 

Another issue with the driving behavior of the platoon, as well as other vehicle types, is the lack of a more sophisticated rule set regarding the incentive to make an overtaking maneuver and what safety criteria should be implemented. There were many situations of overtaking maneuvers by a car that would move too early to the right lane and force the platoon to slow down, leading to strong compression of the platoon. Furthermore, the platoon formation takes too much time to find its steady state in an environment with no other vehicles and constant curvatures. In this approach, the platoon of five members would take 15 seconds to achieve a steady state, which is contrary to reality. It is expected that motorcyclists can establish a fixed formation in a fixed environment much faster.

Finally, one of the assumptions made in this bachelor thesis was that all vehicles are of equal size and occupy only one tile. This assumption fails to take into account the fact that cyclists or motorcyclists may weave through cars in a traffic jam, rather than waiting in a queue. Therefore, this approach may not be appropriate for studying the behavior of platoons in a traffic jam, as in the study by Meng et al. \cite{MENG2007470}.

Modeling platoon behavior in a mixed traffic environment is a challenging task, as there are numerous parameters to consider, such as the number of lanes on a street, vehicle density, and especially vehicle behavior. A cellular automata approach can be used to model platoon behavior and investigate microscopic changes in vehicle behavior and their impact on emergent phenomena in traffic, such as traffic flow. However, due to the numerous assumptions that need to be made and the simplified nature of the model, comparisons between different models and their results can be difficult. Furthermore, there is no reference literature on how to model platoon behavior correctly, and a study of driving behavior requires a separate survey. Overall, the modeling of the fun metric and the assumption that motorcyclists have more fun driving at a certain speed on a curved road supports the hypothesis that roads with more curvature are more enjoyable to drive on.

