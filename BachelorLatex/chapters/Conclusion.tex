\chapter{Conclusion}
\label{chaper:Conclusion}
The bachelor thesis faced several major problems, including the difficulty of gathering real data without GPS data and a deep understanding of vehicle behavior to make the model more realistic. The lack of reference data means that many parameters, such as when a street is congested or how fast one would like to drive, rely mostly on visual evaluation from the simulation. The data gathered from the website "www.curvature.com" also presented challenges, as a deeper understanding of the Keyhole Markup Language and its handling as a data model in a geographic information system is required. Additionally, the data which was gathered from the tooltips instead are very inaccurate and does not provide additional information on how to evaluate the beauty of the trip besides its curvature. 

The fun metric used in this thesis is entirely invented, and there is no reference material to validate its accuracy in reflecting the enjoyment of a motorcyclist. Moreover, modeling motorcyclists with different but fixed preferences regarding distances to each other and speed may result in incompatible motorcyclists and create a matching problem, akin to what is seen in a domino game. Feedback effects where motorcyclists take decisions from other vehicles were completely left out due to their modeling complexity and the lack of available literature on the topic.

The attempt to model the driving behavior solely according to the motorcyclists' preferences as a normal distribution was abandoned due to coding difficulties and logical inconsistencies. For example, when a motorcyclist is in the middle and wants to drive slower so that the motorcyclist behind can catch up, this could conflict with their own desire to catch up to the motorcyclist in front of them. Additionally, modeling the leader like any other member of the platoon showed unrealistic results, as this simulation suggested that the leader has the lowest fun, which might be opposite to reality. The escape velocity, which is an important factor in determining the speed and compression of the platoon could not be researched effectively. In the simulation, all members drove at the same average velocity during the trip making an analysis on the compression of the platoon difficult. 

Another issue with the driving behavior of the platoon, as well as other vehicle types, is the lack of a more sophisticated rule set regarding the incentive to make an overtaking maneuver and what safety criteria should be implemented. There were many situations of overtaking maneuvers by a car that would move too early to the right lane and force the platoon to slow down, leading to strong compression of the platoon. Furthermore, the platoon formation takes too much time to find its steady state in an environment with no other vehicles and constant curvatures. In this approach, the platoon of five members would take 15 seconds to achieve a steady state at best.

Finally, one of the assumptions made in this bachelor thesis was that all vehicles are of equal size and occupy only one tile. This assumption fails to take into account the fact that cyclists or motorcyclists may weave through cars in a traffic jam, rather than waiting in a queue. Therefore, this approach may not be appropriate for studying the behavior of platoons in a traffic jam, as in the study by Meng et al. \cite{MENG2007470}.

Modeling the behavior of platoons in a mixed traffic environment is a challenging task, as it requires consideration of numerous parameters such as the number of lanes on a street, vehicle density, and vehicle behavior. This complexity can make it difficult to compare different models and to gain insights into the behavior of traffic flows in real-world scenarios.

Although this bachelor thesis encountered challenges, the cellular automata model used provides a promising approach to traffic analysis. The model can reflect emergent traffic phenomena and accurately differentiate roads based on a simplified fun metric. By carefully parameterizing the model and incorporating real-world data such as GPS data and survey data on sight preferences, it is possible to analyze local congestion and the enjoyment of curves for a platoon of motorcycles. 
