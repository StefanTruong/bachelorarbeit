\cleardoublepage

% Start with German abstracrt
\begin{otherlanguage}{ngerman}
\chapter*{Kurzfassung}
\addcontentsline{toc}{chapter}{Kurzfassung}

Ziel dieser Arbeit ist es, das Verhalten von Motorradfahrern, die zusammen in einer Kolonne fahren, durch eine auf der Theorie der zellulären Automaten basierende Simulation zu analysieren. Die einschlägige Literatur wird gesichtet, um die Grundlage für den Aufbau der Simulation zu schaffen, und die Anforderungen an die Simulation und den Modellaufbau werden diskutiert. Die Straße wird als zweispurige Straße für eine Richtung modelliert, die in Kacheln unterteilt ist, wobei die Straßenkrümmung die Höchstgeschwindigkeit bestimmt. Der Aufbau ermöglicht einen gemischten Verkehr aus Motorrädern, Autos und Fahrrädern. Eine Kolonne besteht aus Motorradfahrern, die in einer Linienformation mit einem festen Satz von Regeln fahren. Die Simulationsergebnisse einer Kolonne, der auf sieben Straßen im Schwarzwald unterwegs war, werden vorgestellt und analysiert, wobei die Grenzen des Modells anschließend diskutiert werden. Um die Straßen zu bewerten, wird eine Spaßmetrik für Motorräder eingeführt, deren Daten und Krümmungsdefinitionen von \href{www.roadcurvature.com}{roadcurvature.com} stammen. Die Ergebnisse deuten darauf hin, dass Straßen mit höherer Krümmung für Motorradfahrer mehr Spaß bereiten.


\vfill
\noindent\textbf{Stichwörter:} Zelluläre Automaten, Motorradfahrer, Kolonne, Verkehrsfluss, Kurven, Schwarzwald, Spaßfaktor, Mischverkehr
\vfill
\end{otherlanguage}
% Then continue with the english one.
\begin{otherlanguage}{english}
\chapter*{Abstract}
\addcontentsline{toc}{chapter}{Abstract}

This thesis aims to analyze the behavior of motorcyclists driving in a platoon through a simulation based on cellular automata theory. The relevant literature is reviewed to establish the basis for building the simulation, and the requirements for the simulation and model setup are discussed. The road is modeled as a two-lane road for one direction divided into tiles, with the curvature determining the maximum speed. The setup allows for mixed traffic of motorcycles, cars, and bicycles. A platoon consists of motorcyclists riding in a line formation with a fixed set of rules. Simulation results of a platoon traveling on seven roads in the Black Forest region are presented and analyzed, and limitations of the model are discussed. To evaluate the roads, a fun metric for motorcycles is introduced, with data and curvature definitions obtained from \href{www.roadcurvature.com}{roadcurvature.com}. The results suggest that roads with higher curvature are more enjoyable for motorcyclists.

\vfill
\noindent\textbf{Keywords:} cellular automata, motorcycle, platoon, traffic flow, curvature, Black Forest, fun metric, mixed traffic
\vfill
\end{otherlanguage}