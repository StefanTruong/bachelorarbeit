
% set the language here. Last language is the main language. Use `ngerman` (= new german) for German texts.
\usepackage[ngerman, english]{babel}
% \usepackage[english,ngerman]{babel} % If your text mainly is in German.



% general math support
% consider using bracket environments for inline math, i.e. \(x_2^2 + \sqrt{\gamma}\) instead of $$.
% for numbered equations in their own line, use e.g. the array environment. 
\usepackage{amsmath, amssymb}

% bold math package. Set matrices and vectors with \bm{v}
\usepackage{bm}

% beautiful table environments (see https://ctan.org/pkg/booktabs)
\usepackage{booktabs}

% multi page table. Your list of symbols may need this
\usepackage{longtable}

% consistent acronym definitions and usage. You may want to use the glossaries package instead, which is more powerful, but more complex to handle.
%\usepackage[printonlyused, smaller]{acronym}

% for multiple plots in one figue, e.g. Fig 1.a and Fig 1.b
% https://en.wikibooks.org/wiki/LaTeX/Floats,_Figures_and_Captions#Subfloats
\usepackage{subcaption}

% provides \FloatBarrier to prevent floats past some point.
%\usepackage{placeins}

% For vector graphics and MATLAB figures, you may try TikZ:
% There is also tikz-uml for UML diagrams
%\usepackage{tikz}
%\usepackage{pgfplots}

%\pgfplotsset{
%    compat = newest,
%	grid=major,
%	every axis plot/.append style={very thick},
%}

% block diagrams with tikz
%\usetikzlibrary{calc,fit, positioning,arrows.meta}
%\tikzset{>={Latex[width=2mm,length=2mm]}} % more visible default arrow heads
%\tikzstyle{block} = [draw=black, fill=white, rectangle, align=center, minimum height=2em, minimum width=3em]
%\tikzstyle{sum} = [draw, circle, node distance=1cm]

% global matlab2tikz options for exporting MATLAB plots
% https://github.com/matlab2tikz/matlab2tikz
\newlength\figureheight 
\newlength\figurewidth 
\setlength\figureheight{3cm} 
\setlength\figurewidth{0.7\textwidth}


% If you want to use colors, we already defined some for you (university corporate design)
\RequirePackage{xcolor}

\definecolor{UStuttDarkBlue}{RGB}{0,81,158}
\definecolor{UStuttLightBlue}{RGB}{0,190,255}
\definecolor{UStuttDarkGreen}{RGB}{59,140,122}
\definecolor{UStuttLightGreen}{RGB}{125,155,101}
\definecolor{UStuttDarkOrange}{RGB}{228,175,52}
\definecolor{UStuttLightOrange}{RGB}{236,218,145}


% for code listings, you can e.g. use the "listings" package (http://texdoc.net/texmf-dist/doc/latex/listings/listings.pdf):
\usepackage{listings} 
%\usepackage{scrhack} % if you load listings together with scrbook etc., then load this fixing package as well

\lstset{
  captionpos=b,
  commentstyle=\color{UStuttDarkGreen},
  frame=single,	                   % adds a frame around the code
  keepspaces=true,
  %keywordstyle=\color{UStuttDarkBlue},
  showspaces=false,
  showstringspaces=false,          % underline spaces within strings only
  showtabs=false,
  stringstyle=\color{UStuttDarkBlue},
  tabsize=2
}


% This class does the ISW styling for you (together with scrbook).
%
% It handles the following:
% - Proper input and font encoding (Just type, don't care about the LaTeX compiler you use or how to type German umlauts)
% - Fonts with ligatures and kerning (Tex Gyre fonts are used, part of every LaTeX installation, text is nice to read)
% - Bibliography styling for biblatex (declare your bibliography file and you are ready to go)
% - Provide command for title page (\makeISWtitle) and declaration of originality ( \declarationOfOriginality)
% - Loads packages "biblatex" and "graphics"
\usepackage[
    type=bachelor, % bachelor, study, bachelorproject
]{iswthesis}

% hyperref provides hyperlinks within the document, but also auto-naming.
% E.g. when referencing, instead of typing "Figure~\ref{fig:XY}" try "\autoref{fig:XY}".
% You may want to use `clevceref` instead of using \autoref in the hyperref package, which has slightly more possibilities.
\PassOptionsToPackage{pdfpagelabels}{hyperref}
\usepackage{hyperref}  % backref linktocpage pagebackref
\pdfcompresslevel=9
\pdfadjustspacing=1

\hypersetup{%
    %draft, % = no hyperlinking at all
    %colorlinks=true,
    colorlinks=false, 
    linktocpage=false, pdfborder={0 0 0},%
    breaklinks=true, pdfpagemode=UseNone, pageanchor=true, pdfpagemode=UseOutlines,%
    plainpages=false, bookmarksnumbered, bookmarksopen=true, bookmarksopenlevel=1,%
    hypertexnames=true, pdfhighlight=/O,%nesting=true,%frenchlinks,%
    %urlcolor=Black, linkcolor=Black, citecolor=Black, %pagecolor=Black,%
} 


% Your own commands (https://en.wikibooks.org/wiki/LaTeX/Macros):
% Consider defining your own commands for often used terms, e.g.

% Real numbers symbol
\newcommand{\R}{\mathbb{R}}

% Transpose of vector or matrix (upright)
\newcommand{\T}{\mathrm{T}}

\newcommand{\mustbe}{\ensuremath{\stackrel{!}{=}}}

% short matrix environment. Instead of typing \begin{bmatrix} 1 & 2 \\ 3 & 4 \end{bmatrix} you can now use as well \bmat{1 & 2 \\ 3 & 4}
\newcommand{\bmat}[1]{ \ensuremath{\begin{bmatrix} #1 \end{bmatrix}} }

% partial derivative: \partfrac{^2}{x^2} yields ∂²/∂x²
\newcommand{\partfrac}[2]{ \ensuremath{\frac{\partial #1}{\partial #2}} }

% upright "d" for differentiation
\newcommand{\ddiff}{\ensuremath{\mathrm{d}}}

% d/dt
\newcommand{\ddt}{\ensuremath{\frac{\ddiff}{\ddiff t}}}

